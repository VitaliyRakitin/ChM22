\section{Аппроксимация}

Рассматривается уравнение переноса:
\[
\frac{\partial u}{\partial t} - \frac{1}{2}
\frac{\partial u}{\partial x} = 0.
\] 

\begin{enumerate}
\item Посчитаем апроксимацию для явной схемы (\ref{one1})

\[
v_m^{n+1} - \frac{1}{2} \left( \left(1 + \frac{\tau}{h} \right)v^n_{m+1} + \left(1 - \frac{\tau}{h} \right)v^n_{m-1}  \right) =
\]
\[
= v_m^n + \tau {\dot v}_m^n + O(\tau^2) - \frac{1}{2} \left( \left(1 + \frac{\tau}{h} \right)\left(v_m^n + h {v'}_m^n + O(h^2)\right) +\left(1 - \frac{\tau}{h} \right) \left(v_m^n - h {v'}_m^n + O(h^2) \right)  \right) = 
\]
\[
 = \tau \left({\dot v}_m^n    - \frac{1}{2}  {v'}_m^n \right) + O(\tau^2+h^2) = O(\tau^2+h^2).
\]

\item Для неявной схемы (\ref{two1})

\[
4h^2 v_m^n + \tau
\left(\omega \tau -h \right) v_{m-1}^{n+1} -
2\left(2h^2 + \omega \tau^2\right) v_{m}^{n+1} +
\tau \left( \omega \tau + h \right) v_{m+1}^{n+1} = 
\]
\[
=4h^2 v_m^n
+  \tau \left(\omega \tau -h \right) \left(v_m^n+ \tau {\dot v}_m^n - h {v'}_m^n - \tau h {\dot v'}_m^n + O(h^2 + \tau^2)\right)
- 2\left(2h^2 + \omega \tau^2\right) \left( v_m^n+ \tau{\dot v}_m^n + O(\tau^2)\right) +\]
\[
+ \tau \left( \omega \tau + h \right) \left(v_m^n+ \tau {\dot v}_m^n + h {v'}_m^n + \tau h {\dot v'}_m^n + O(h^2 + \tau^2)\right)
= 
\]
\[
=\cancel{4h^2 v_m^n}
+  \omega \tau^2  \left(v_m^n+ \tau {\dot v}_m^n - \cancel{h {v'}_m^n - \tau h {\dot v'}_m^n} + O(h^2 + \tau^2)\right)
- \tau h   \left(\cancel{v_m^n+ \tau {\dot v}_m^n}   -  h {v'}_m^n - \tau h {\dot v'}_m^n + O(h^2 + \tau^2)\right) - 
\]
\[
- 4h^2 \left( \cancel{v_m^n} + \tau{\dot v}_m^n + O(\tau^2)\right) + 2\omega \tau^2\left( v_m^n+ \tau{\dot v}_m^n + O(\tau^2)\right) 
+ \omega \tau^2 \left(v_m^n+ \tau {\dot v}_m^n + \cancel{h {v'}_m^n + \tau h {\dot v'}_m^n} + O(h^2 + \tau^2)\right) +
\]
\[
+
\tau h \left(\cancel{v_m^n+ \tau {\dot v}_m^n} + h {v'}_m^n + \tau h {\dot v'}_m^n + O(h^2 + \tau^2)\right)
= O(\tau^2 + \tau h^2)
\]

%\[
%=
%+  \omega \tau^2  \left(v_m^n+ \tau {\dot v}_m^n + O(h^2 + \tau^2)\right)
%- \tau h   \left( -  h {v'}_m^n - \tau h {\dot v'}_m^n + O(h^2 + \tau^2)%\right) - 
%\]
%\[
%- 4h^2 \left( \tau{\dot v}_m^n + O(\tau^2)\right) + 2\omega \tau^2\left( v_m^n+ \tau{\dot v}_m^n + O(\tau^2)\right) +
%\]
%\[
%+ \omega \tau^2 \left(v_m^n+ \tau {\dot v}_m^n  + O(h^2 + \tau^2)\right) +
%\tau h \left( h {v'}_m^n + \tau h {\dot v'}_m^n + O(h^2 + \tau^2)\right)
%= 
%\]
\end{enumerate}



\section{Дифференциальное приближение}
\begin{enumerate}
\item Дифференциальное приближение для явной схемы  (\ref{one}) с точностью до членов порядка $O\left(\tau^3+ h^3 +\frac{h^4}{\tau}\right) $
\[
v_t - \frac{1}{2} v_{ \overset{\circ}{x} } = \frac{h^2}{2\tau} v_{x \ol{x}}
\]


\[
\dot v + \frac{\tau}{2} \ddot v  + \frac{\tau^2}{6} \dddot v + O\left(\tau^3\right) - \frac{1}{2} v' -  \frac{h}{4}v'' - \frac{h^2}{12} v''' + O\left(h^3\right) = 
\frac{h^2}{2\tau} v'' + \frac{h^3}{4\tau} v''' + O\left(\frac{h^4}{\tau}\right) 
\]
\[
\dot v - \frac{1}{2} v' - \frac{h^2}{2\tau} v'' = - \frac{\tau}{2} \ddot v  - \frac{\tau^2}{6} \dddot v  +  \frac{h}{4}v'' + \frac{h^2}{12} v''' 
 + \frac{h^3}{4\tau} v'''' + O\left(\tau^3+ h^3 +\frac{h^4}{\tau}\right) 
\]
\[
\dot v = \frac{1}{2} v' + O\left(\tau + h + \frac{h^2}{\tau}\right)
\]
\[
\dot v' = \frac{1}{2} v'' + O\left(\tau + h + \frac{h^2}{\tau}\right)
\]
\[
\dot v'' = \frac{1}{2} v''' + O\left(\tau + h + \frac{h^2}{\tau}\right)
\]
\[
\ddot v = \frac{1}{2} \dot v' + O\left(\tau + h + \frac{h^2}{\tau}\right) = \frac{1}{4} v'' +  O\left(\tau + h + \frac{h^2}{\tau}\right) 
\]
\[
\dddot v = \frac{1}{4} \dot v'' + O\left(\tau + h + \frac{h^2}{\tau}\right) = \frac{1}{8} v''' +  O\left(\tau + h + \frac{h^2}{\tau}\right) 
\]


\[
\dot v - \frac{1}{2} v' - \frac{h^2}{2\tau} v'' = \left(\frac{h}{4} - \frac{\tau}{8} \right) v''  + \left(\frac{h^2}{12}  - \frac{\tau^2}{48} \right)  v'''   + \frac{h^3}{4\tau} v'''' + O\left(\tau^3+ h^3 +\frac{h^4}{\tau}\right) 
\]

\item Дифференциальное приближение для явной схемы  (\ref{two}) с точностью до членов порядка $O\left(\tau^3 +h^3 + \tau h^2\right)$
\[
v_t - \frac{1}{2}\hat{v}_{ \overset{\circ}{x} } = \frac{\omega \tau}{4} \hat{v}_{x \ol{x}}
\]


\[
\dot v + \frac{\tau}{2} \ddot v  + \frac{\tau^2}{6} \dddot v + O\left(\tau^3\right) 
- \frac{1}{2} v' -  \frac{h}{4}v'' - \frac{h^2}{12} v''' + O\left(h^3\right) = 
 \frac{\omega \tau}{4} v'' +  \frac{\omega \tau h}{8} v''' + O\left(\tau h^2\right) 
\]
\[
\dot v - \frac{1}{2} v' +  \frac{\omega \tau}{4} v'' =
- \frac{\tau}{2} \ddot v  - \frac{\tau^2}{6} \dddot v 
 +  \frac{h}{4}v'' + \frac{h^2}{12} v''' 
 +  \frac{\omega \tau h}{8} v''' + 
 O\left(\tau^3 +h^3 + \tau h^2\right)
\]
\[
\dot v = \frac{1}{2} v' + O\left( \tau + h \right)
\]
\[
\dot v' = \frac{1}{2} v'' + O\left( \tau + h \right)
\]
\[
\dot v'' = \frac{1}{2} v''' + O\left( \tau + h \right)
\]
\[
\ddot v = \frac{1}{2} \dot v' + O\left( \tau + h \right) = 
\frac{1}{4} v'' + O\left( \tau + h \right)
\]
\[
\dddot v = \frac{1}{4} \dot v'' + O\left( \tau + h \right) = 
\frac{1}{8} v''' + O\left( \tau + h \right)
\]
\[
\dot v - \frac{1}{2} v' +  \frac{\omega \tau}{4} v'' =
 \left(\frac{h}{4}
- \frac{\tau}{8} \right) v''  
  + \left(\frac{h^2}{12} 
 +  \frac{\omega \tau h}{8} - \frac{\tau^2}{48} \right) v'''  + 
 O\left(\tau^3 +h^3 + \tau h^2\right)
\]
\end{enumerate}

\section{Устойчивость}

\begin{enumerate}
\item Для явной схемы (\ref{one1})
\[
v_{m}^{n+1}  = \frac{1}{2}\left( \left(1 + \frac{\tau}{h}\right) v_{m+1}^n  + \left(1 - \frac{\tau}{h}\right) v_{m-1}^n \right)
\]
Сделаем замену
\[
v_m^n = 
(\lambda(\phi))^n e^{im\phi}
\]
Тогда
\[
\lambda^{n+1} e^{im\phi} = \frac{1}{2}\left( \left(1 + \frac{\tau}{h}\right) \lambda^{n} e^{i(m+1)\phi}  + \left(1 - \frac{\tau}{h}\right) \lambda^{n} e^{i(m-1)\phi}  \right)
\]
Сократим всё на $\lambda^{n} e^{im\phi} $
\[
\lambda = \frac{1}{2}\left( \left(1 + \frac{\tau}{h}\right) e^{i\phi}  + \left(1 - \frac{\tau}{h}\right) e^{-i\phi}  \right)
\]
\[
\lambda = \frac{1}{2}\left( e^{i\phi} +  e^{-i\phi} + \frac{\tau}{h} \left(e^{i\phi}  -  e^{-i\phi} \right) \right)
\]
\[
\lambda = \cos\phi + \frac{i\tau}{h} \sin\phi
\]
\[
| \lambda | = \sqrt{\cos^2\phi +  \frac{\tau^2}{h^2} \sin^2\phi} = 
 \sqrt{1 +  (\frac{\tau^2}{h^2} - 1) \sin^2\phi} \le 1 \quad \Longleftrightarrow \quad -1 \le \frac{\tau^2}{h^2} - 1 \le 0
\]
\[
0 \le \frac{\tau^2}{h^2} \le 1 \qquad \Longrightarrow \qquad  \tau^2 \le h^2 \qquad \Longrightarrow \qquad \tau \le h. 
\]
Таким образом, можем сделать вывод, что наша схема устойчива при условии $\tau \le h$.
\item Для неявной схемы (\ref{two1})
\[
4h^2 v_m^n + \tau
\left(\omega \tau -h \right) v_{m-1}^{n+1} -
2\left(2h^2 + \omega \tau^2\right) v_{m}^{n+1} +
\tau \left( \omega \tau + h \right) v_{m+1}^{n+1}= 0
\]
Аналогично предыдущему пункту сделаем замену
\[
v_m^n = 
(\lambda(\phi))^n e^{im\phi}
\]
Тогда
\[
4h^2 \lambda^{n} e^{im\phi} + \tau
\left(\omega \tau -h \right) \lambda^{n+1} e^{i(m-1)\phi} -
2\left(2h^2 + \omega \tau^2\right) \lambda^{n+1} e^{im\phi} +
\tau \left( \omega \tau + h \right) \lambda^{n+1} e^{i(m+1)\phi}= 0
\]
Сократим всё на $\lambda^{n+1} e^{im\phi} $
\[
\frac{4h^2}{\lambda} + \tau
\left(\omega \tau -h \right)  e^{-i\phi} -
2\left(2h^2 + \omega \tau^2\right) +
\tau \left( \omega \tau + h \right)  e^{i\phi}= 0
\]
\[
\frac{4h^2}{\lambda} + \omega \tau^2 
\left(  e^{-i\phi} + e^{i\phi} \right)  -
2\left(2h^2 + \omega \tau^2\right) +
\tau h \left(  e^{i\phi} - e^{-i\phi} \right)   = 0
\]
\[
\frac{4h^2}{\lambda} + 2\omega \tau^2 
\cos\phi  -
2\left(2h^2 + \omega \tau^2\right) +
2 i\tau h \sin\phi  = 0
\]
\[
\frac{4h^2}{\lambda} + 2\omega \tau^2 
\cos\phi  -
2\left(2h^2 + \omega \tau^2\right) +
2 i\tau h \sin\phi  = 0
\]
\[
-1 \le \lambda = \frac{2h^2}{2h^2 + \omega \tau^2 - \omega \tau^2 
\cos\phi  -  i\tau h \sin\phi} \le 1
\]
Рассмотрим правый случай
\[
2h^2 \le 2h^2 + \omega \tau^2 - \omega \tau^2 
\cos\phi  -  i\tau h \sin\phi 
\]
\[
\omega \tau^2 
\cos\phi  +  i\tau h \sin\phi \le \omega \tau^2 
\]
\[
\cos\phi  +  \frac{i h}{\tau \omega} \sin\phi \le 1
\]
\[
\cos^2\phi  +  \frac{h^2}{\tau^2 \omega^2} \sin^2\phi \le 1
\]
\[
1  + \left(\frac{h^2}{\tau^2 \omega^2} - 1\right) \sin^2\phi \le 1
\]
\[
\frac{h^2}{\tau^2 \omega^2} \le 1  \qquad \Longrightarrow \qquad
h \le \tau \omega
\]
Рассмотрим левый случай
\[
-2h^2 \le 2h^2 + \omega \tau^2 - \omega \tau^2 
\cos\phi  -  i\tau h \sin\phi 
\]
\[
  \omega \tau^2 
\cos\phi  +  i\tau h \sin\phi \le 4h^2 + \omega \tau^2 
\]
\[
\cos\phi  +  \frac{i h}{\tau \omega}  \sin\phi \le \frac{4h^2}{\omega \tau^2 } + 1
\]
\[
\cos^2\phi  +  \frac{h^2}{\tau^2 \omega^2}  \sin^2\phi \le \left(\frac{4h^2}{\omega \tau^2 } + 1\right)^2
\]
\[
1  +  \left(\frac{h^2}{\tau^2 \omega^2} - 1\right)  \sin^2\phi \le \frac{16h^4}{\omega^2 \tau^4 } + \frac{8h^2}{\omega \tau^2 } + 1
\]
\[
\left(\frac{h^2}{\tau^2 \omega^2} - 1\right)  \sin^2\phi \le \frac{16h^4}{\omega^2 \tau^4 } + \frac{8h^2}{\omega \tau^2 } 
\]
Так как мы уже выяснили, что
\[
\frac{h^2}{\tau^2 \omega^2} \le 1
\]
Получим
\[
|\sin\phi| \ge \sqrt{\frac{\frac{16h^4}{\omega^2 \tau^4 } + \frac{8h^2}{\omega \tau^2}}{1 - \frac{h^2}{\tau^2 \omega^2}}}
\]
Рассмотрим $\phi = \frac{\pi}{2}$
\[
1 \ge \frac{\frac{16h^4}{\omega^2 \tau^4 } + \frac{8h^2}{\omega \tau^2}}{1 - \frac{h^2}{\tau^2 \omega^2}}
\]
\[
\frac{16h^4}{\omega^2 \tau^4 } + \frac{8h^2}{\omega \tau^2} +  \frac{h^2}{\tau^2 \omega^2} - 1 \le 0
\]
\[
\frac{16h^4}{ \tau^4 } + (8\omega + 1) \frac{h^2}{\tau^2 } - \omega^2 \le 0
\]
Рассморим случай 
\[
\frac{16h^4}{ \tau^4 } + (8\omega + 1) \frac{h^2}{\tau^2 } - \omega^2 = 0
\]
и делаем замену $x = \frac{h^2}{\tau^2 } $
\[
16x^2 + (8\omega + 1) x - \omega^2 = 0
\]
\[
D =  (8\omega + 1)^2 + 4 \cdot 16 \cdot \omega^2 = 128 \omega^2 + 16 \omega + 1
\]
\[
x = \frac{- 8 \omega - 1 \pm \sqrt{128 \omega^2 + 16 \omega + 1}}{32}
\]
Значит 
\[
\frac{h^2}{\tau^2} \in \left[ \frac{- 8 \omega - 1 - \sqrt{128 \omega^2 + 16 \omega + 1}}{32},  \frac{- 8 \omega - 1 + \sqrt{128 \omega^2 + 16 \omega + 1}}{32}  \right]
\]
При $\omega = 1$
\[
\frac{h^2}{\tau^2} \in \left( 0,  0.095 \right)
\]
\[
\frac{h}{\tau} \in \left( 0,  0.3 \right)
\]
При $\omega = 0.1$
\[
\frac{h^2}{\tau^2} \in \left( 0,  0.0053 \right)
\]
\[
\frac{h}{\tau} \in \left( 0,  0.073 \right)
\]
Таким образом, мы можем подобрать такие $\tau$ и $h$, чтобы наша схема была устойчива в точке $\phi = \frac{\pi}{2}$, а~значит она условно устойчива.
\end{enumerate}
