\newpage
\section{Расчёт нелинейной задачи}

\subsection{Постановка задачи}
\[
F(u) = -\frac{1}{2} u^2
\]

Аналагично линейному случаю рассматриваются явная и неявная схемы (\ref{one}) и (\ref{two}) для численного решения уравнения  
\[
\frac{\partial u}{\partial t} - \frac{1}{2} 
\frac{\partial u^2}{\partial x} = 0
\]
в области (\ref{line}).  

\subsubsection{Граничные условия}
\begin{enumerate}
\item на прямой $t = 0$ (при $-1 \le x \le 1$) определена функция (\ref{gran});
\item на прямой $x = -1$, ввиду постоянства решения на характеристиках (см.\ref{harr_nonlin}) "--- прямых вида 
\[
x = -\phi(\xi)t + \xi; \qquad
\phi(\xi) = 
\begin{cases}
0,& \text{если} \quad \xi \le 0, \\
4\xi,& \text{если} \quad 0 < \xi \le 0.25, \\
1,& \text{если} \quad \xi > 0.25.
\end{cases}
\]
Значит,
\[
u(t,-1) = 0;
\]
\item на прямой $x = 1$, аналогично
\[
u(t,1) = 1.
\] 
\end{enumerate}

\subsection{Преобразование схем}
Преобразуем уравнение (\ref{one}) к более удобному виду
\[
\frac{v_{m}^{n+1} - v_m^n}{\tau} -  \frac{\left(v^2\right)_{m+1}^n -  \left(v^2\right)_{m-1}^n}{4h} = \frac{h^2}{2\tau} \frac{v_{m+1}^n - 2v_{m}^n + v_{m-1}^n}{h^2} 
\]

\begin{align}\label{one2}
\boxed{
 v_{m}^{n+1}  =  \frac{\tau}{2h} \left( \left(v^2\right)_{m+1}^n -   \left(v^2\right)_{m-1}^n \right) + \frac{1}{2}v_{m+1}^n  + \frac{1}{2}v_{m-1}^n 
}
\tag{\ref{one}**}
\end{align}

Преобразуем уравнение (\ref{two}) к более удобному виду
\[
\left(F'_v(v)\right)^2 = \left(\left(-\frac{1}{2}v^2\right)'_v \right)^2 = v^2	
\]

\begin{equation} \notag
\begin{split}
\left(v^2\hat{v}_x\right)_{\ol{x}} &= 
\left(\left(v^2\right)_{m}^{n}\left(v_{m}^{n+1}\right)_x\right)_{\ol{x}} = 
\frac{\left(v^2\right)_{m}^{n}(v_x)_{m}^{n+1} - \left(v^2\right)_{m-1}^{n}(v_x)_{m-1}^{n+1}}{h} = \\ 
&= \frac{\left(v^2\right)_{m}^{n}v_{m+1}^{n+1} 
- \left(\left(v^2\right)_{m}^{n} + \left(v^2\right)_{m-1}^{n}\right)v_{m}^{n+1} 
+ \left(v^2\right)_{m-1}^{n}v_{m-1}^{n+1}}{h^2} 
\end{split}
\end{equation}

\[
(F(\hat{v}))_{ \overset{\circ}{x} } = 
-\frac{1}{2}\hat{v}_{ \overset{\circ}{x}}^2 =
-\frac{\left(v^2\right)_{m+1}^{n+1} - \left(v^2\right)_{m-1}^{n+1}}{4h}
\]

Соответственно,
\[
\frac{v_{m}^{n+1} - v_m^n}{\tau}  -\frac{\left(v^2\right)_{m+1}^{n+1} - \left(v^2\right)_{m-1}^{n+1}}{4h} = \omega \tau \left(\frac{\left(v^2\right)_{m}^{n}v_{m+1}^{n+1} 
- \left(\left(v^2\right)_{m}^{n} + \left(v^2\right)_{m-1}^{n}\right)v_{m}^{n+1} 
+ \left(v^2\right)_{m-1}^{n}v_{m-1}^{n+1}}{h^2}  
\right)
\]

\begin{equation} \label{two2}
\boxed{
\begin{split}
4h^2\left( v_{m}^{n+1} - v_m^n \right) &-
\tau h\left(\left(v^2\right)_{m+1}^{n+1} - \left(v^2\right)_{m-1}^{n+1}\right) - \\
&- 4 \omega \tau^2 \left(\left(v^2\right)_{m}^{n}v_{m+1}^{n+1} 
- \left(\left(v^2\right)_{m}^{n} + \left(v^2\right)_{m-1}^{n}\right)v_{m}^{n+1} 
+ \left(v^2\right)_{m-1}^{n}v_{m-1}^{n+1}\right)
 = 0
\end{split}
}
\tag{\ref{two}**}
\end{equation}



Нормируем схему для использования метода прогонки

\[
\begin{split}
-4h^2 v_m^n 
+\left(4 \omega \tau^2\left(v^2\right)_{m-1}^{n}
v_{m-1}^{n+1} +
\tau h\left(v^2\right)_{m-1}^{n+1}\right) +
 \left(4h^2  + 4 \omega \tau^2  \left(\left(v^2\right)_{m}^{n} + \left(v^2\right)_{m-1}^{n}\right) \right) v_{m}^{n+1} 
+\\
+ \left( 4 \omega \tau^2 \left(v^2\right)_{m}^{n}
v_{m+1}^{n+1} 
- \tau h\left(v^2\right)_{m+1}^{n+1} \right)
 = 0
\end{split}
\]
Для упрощения внешнего вида схемы заменим все константы на конкретном шаге вычисления на 
\[
A[M] = \frac{4 \omega \tau^2 \left(v^2\right)_{M}^{n}}{4h^2  + 4 \omega \tau^2  \left(\left(v^2\right)_{m}^{n} + \left(v^2\right)_{m-1}^{n}\right)}, \quad B = \frac{\tau h}{4h^2  + 4 \omega \tau^2  \left(\left(v^2\right)_{m}^{n} + \left(v^2\right)_{m-1}^{n}\right)}, \quad C = \frac{-4h^2}{4h^2  + 4 \omega \tau^2  \left(\left(v^2\right)_{m}^{n} + \left(v^2\right)_{m-1}^{n}\right)}
\]
Тогда наша схема принимает вид
\[
C v_m^n 
+\left(A[m-1] v_{m-1}^{n+1} +
B \left(v^2\right)_{m-1}^{n+1}\right) +
v_{m}^{n+1} 
+ \left(A[m]
v_{m+1}^{n+1} 
- B \left(v^2\right)_{m+1}^{n+1} \right)
 = 0
\]

Теперь получим систему вида $G_i = G_i(\hat{v})$:

\[
\begin{cases}
G_0 = Cv_0 +
\hat{v}_{0}
+ \left(A[0]
\hat{v}_{1}
- B \hat{v}^2_{1} \right), & m = 0;\\
G_m = C v_m
+\left(A[m-1] \hat{v}_{m-1} +
B \hat{v}^2_{m-1}\right) +
\hat{v}_{m} 
+ \left(A[m]
\hat{v}_{m+1}
- B \hat{v}^2_{m+1} \right),
 & m = 1,\dots, M - 1;\\
G_M = C v_M
+\left(A[M-1] \hat{v}_{M-1} +
B \hat{v}^2_{M-1}\right) +
\hat{v}_{M}, & m = M.
\end{cases}
\]

Найдём матрицу Якоби

\[
J =  \begin{pmatrix}
        1 &  A[0] - 2B\hat{v}_1 & 0 &\dots & 0  \\
        A[0] + B\hat{v}_0  & 1 & A[1] - 2B\hat{v}_2 & 0 & \dots \\
        0 & A[1] + B\hat{v}_1  & 1 & A[2] - 2 B\hat{v}_3 &  \dots  \\
		\vdots& & \ddots&\ddots &\ddots  \\
        0 &  \dots & A[M-2] + B\hat{v}_{M-2}  & 1 & A[M-1] -2 B\hat{v}_{M}  \\
        0 &  \dots & 0 & A[M-1] + B\hat{v}_{M-1} & 1\\
    \end{pmatrix}
\]