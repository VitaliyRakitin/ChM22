\newpage
\subsection{Характеристики}

\subsubsection{Лиинейный случай}
\[
\frac{\partial u}{\partial t} - \frac{1}{2}
\frac{\partial u}{\partial x} = 0
\]

Составим характеристическую систему
\[
\frac{dt}{1} = \frac{dx}{ - \frac{1}{2}} 
\]

\begin{equation}\label{harr_lin}
t = -2x + C_1.
\end{equation}


\subsubsection{Нелинейный случай}
\[
\frac{\partial u}{\partial t} - \frac{1}{2}
\frac{\partial u^2}{\partial x} = 0
\]

\[
\frac{\partial u}{\partial t} - u
\frac{\partial u}{\partial x} = 0
\]

Составим характеристическую систему
\[
\frac{dt}{1} = \frac{dx}{-u} = \frac{du}{0}
\]

Значит, получим уравнения
\[
u = C_1, \qquad x + u t = C_2
\]
начальное значение (\ref{gran}):
значит 
\[
C_2 =  \sigma;\qquad
C_1 = \phi(\xi) = u_0(\xi)= 
\begin{cases}
0,& \text{если} \quad \xi \le 0, \\
4\xi,& \text{если} \quad 0 < \xi \le 0.25, \\
1,& \text{если} \quad \xi > 0.25;
\end{cases}
\]

Тогда
\begin{equation}\label{harr_nonlin}
x + \phi(\xi)t = \xi \quad \Longrightarrow \quad x = -\phi(\xi)t + \xi.
\end{equation}